\documentclass{article}

\title{STAT 2910 SAMPLE MIDTERM}

\begin{document}

\maketitle

\section*{Chapter 4}
\section*{Probability and Probability Distributions}

\begin{enumerate}
    \item An experiment involves tossing a single die. These are some events:
    
    A: Observe a 2

    B: Observe an even number

    C: Observe a number greater than 2

    D: Observe both A and B

    E: Observe A or B or both

    F: Observe both A and C
    
\paragraph*{Question Key} 

    
    
    \hfill \break a. List the simple events in the sample space

    b. List the simple events in each of the events A through F

    c. What probabilities should you assign to the simple events

    d. Calculate the probabilities of the six events by adding the appropriate simple-event probabilities
    
    \item A sample space S consists of five simple events with these probabilities:
    $$P(E_1) = P(E_2) = 0.15$$
    $$P(E_3) = 0.4$$
    $$P(E_4) = 2P(E_5)$$
    
     a. Find the probability for the simple events $E_4$ and $E_5$
    
     b. Find the probabilities for these two events:
        $$A = (E_1, E_3, E_4)$$
        $$B = (E_2, E_3)$$
   
    c. List the simple events that are either in event A or event B or both

    d. List the simple events that are either in both event A and event B
    
    \item A sample space contains 10 simple events: $E_1, E_2,...,E_10$. If $P(E_1) = 3P(E_2) = 0.45$ and the remaining simple events are equiprobable, find the probabilities of these remaining simple events
    \item A basketball player hits 70\% of her free throws. When she tosses a pair of free throws, the four possible simple events and three of their associated probabilities are given in the following list:
    
    \begin{center}
    \begin{tabular}{|c|c|c|c|c|}
        \hline
        Simple Events & Outcome1 & Outcome2 & Probability1 \\
        1 & Hit & Hit & 0.49 \\
        2 & Hit & Miss & ? \\
        3 & Miss & Hit & 0.21 \\
        4 & Miss & Miss & 0.09 \\
        \hline
    \end{tabular}
\end{center}
a. Find the probability that the player will hit on the first throw and miss on the second.

b. Find the probability that the player will hit on at least one of the two free throws.

\item A bowl contains three red and two yellow balls. Two balls are randomly selected and their colors recorded. Use a tree diagram to list the 20 simple events in the experiment, keeping in mind the order in which the balls are drawn.

\item Refer to Exercise 4.7. A ball is randomly selected from the bowl containing three red and two yellow balls. Its color is noted, and the ball is returned to the bowl before a second ball is selected. List the additional five simple events that must be added to the sample space in Exercise 4.7.


\item Four equally qualified runners, John, Bill, Ed, and Dave, run a 100-meter sprint, and the order of finish is recorded.

a. How many simple events are in the sample space?

b. If the runners are equally qualified, what probability should you assign to each simple event?

c. What is the probability that Dave wins the race?

d. What is the probability that Dave wins and John places second?

e. What is the probability that Ed finishes last?

\item You have two groups of distinctly different items, 10 in the first group and 8 in the second. If you select one item from each group, how many different pairs can you form?
\item Three dice are tossed. How many simple events are in the sample space?


\item Evaluate the following permutations. \\

 a. $P_3^5$    b. $P_9^{10}$    c. $P_6^6$    d. $P_1^{20}$ \\

\item Evaluate the following combinations. \\

 a. $C_3^5$    b. $C_9^{10}$    c. $C_6^6$    d. $C_1^{20}$ \\

\item A student prepares for an exam by studying a list of 10 problems. She can solve 6 of them. For the exam, the instructor selects 5 questions at random from the list of 10. What is the probability that the student can solve all 5 problems on the exam?

\item A monkey is given 12 blocks: 3 shaped like squares, 3 like rectangles, 3 like triangles, and 3 like circles. If it draws three of each kind in order—say, 3 triangles, then 3 squares, and so on—would you suspect that the monkey associates identically shaped figures? Calculate the probability of this event.

\item Five cards are selected from a 52-card deck for a poker hand. \\ 
a. How many simple events are in the sample space?

b. A royal flush is a hand that contains the A, K, Q, J, and 10, all in the same suit. How many ways are there to get a royal flush?

c. What is the probability of being dealt a royal flush?

\item Refer to the previous exercise. You have a poker hand containing four of a kind. \\ 
a. How many possible poker hands can be dealt?

b. In how many ways can you receive four cards of the same face value and one card from the other 48 available cards?

c. What is the probability of being dealt four of a kind?

\item An experiment can result in one of five equally likely simple events $E_1,E_2,...E_5.$ Events A, B and C are defined as follows: \\ 
$$ A:E_1,E_3 \:\:\:\:\:\:\:\: P(A) = 0.4 $$
$$ A:E_1,E_2,E_4,E_5 \:\:\:\:\: P(B) = 0.8 $$
$$ A:E_3,E_4 \:\:\:\:\:\:\:\: P(C) = 0.4 $$
Find the probabilities associated with these compound events by listing the simple events in each. \\
a. $A^C$

b. $A \cap B$

c. $B \cap C$









\end{enumerate}


\end{document}





